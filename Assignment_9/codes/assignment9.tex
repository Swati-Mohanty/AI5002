\documentclass[journal,12pt,twocolumn]{IEEEtran}
\usepackage{amsthm}
\usepackage{graphicx}
\usepackage{mathrsfs}
\usepackage{txfonts}
\usepackage{stfloats}
\usepackage{pgfplots}
\usepackage{cite}
\usepackage{cases}
\usepackage{mathtools}
\usepackage{caption}
\usepackage{enumerate}	
\usepackage{enumitem}
\usepackage{amsmath}
\usepackage[utf8]{inputenc}
\usepackage[english]{babel}
\usepackage{multicol}
%\usepackage{xtab}
\usepackage{longtable}
\usepackage{multirow}
%\usepackage{algorithm}
%\usepackage{algpseudocode}
\usepackage{enumitem}
\usepackage{mathtools}
\usepackage{gensymb}
\usepackage{hyperref}
%\usepackage[framemethod=tikz]{mdframed}
\usepackage{listings}
    %\usepackage[latin1]{inputenc}                                 %%
    \usepackage{color}                                            %%
    \usepackage{array}                                            %%
    \usepackage{longtable}                                        %%
    \usepackage{calc}                                             %%
    \usepackage{multirow}                                         %%
    \usepackage{hhline}                                           %%
    \usepackage{ifthen}                                         %%
  \providecommand{\nCr}[2]{\,^{#1}C_{#2}}
  \providecommand{\nPr}[2]{\,^{#1}P_{#2}}
  \lstset{
%language=C,
frame=single, 
breaklines=true,
columns=fullflexible
}

\title{Assignment 9
\\Probability and Random Variables }
\author{Swati Mohanty (EE20RESCH11007) }
\date{April 2021}

\begin{document}

\maketitle


\section{Problem}
Given : X (t ) is a random process with a constant mean value of 2 and the auto
correlation function 
\begin{equation*}
   R_{xx}(\tau) = 4(e^{-0.2|\tau|}+1) 
\end{equation*}
\\Let X be the Gaussian random variable obtained by sampling the process at $t=t_i$ and let 
\begin{equation*}
Q(\alpha)={\int_{\alpha}^{\infty}}\dfrac{-1}{\sqrt{2\pi}}e^{\frac{-y^2}{2}}dy
\end{equation*}
%
The probability that $[\textit{X}\leqslant1]$ is

\section{Solution}
Given auto correlation function is
\begin{equation*}
   R_{xx}(\tau) = 4(e^{-0.2|\tau|}+1) 
\end{equation*}
At X = 0
\begin{align}
     R_{xx}(0) = 4(e^{0}+1) = 8
     \\\implies \sigma^2 = 8
    \\ \implies \sigma = 2\sqrt{2} ; \mu = 0{}{}
\end{align}
Now P($[\textit{X}\leqslant1]$) = $F_x(1)$
\begin{align}
    =1 - Q\bigg(\frac{X-\mu}{\sigma}\bigg)
    \\=1 - Q\bigg(\frac{1-0}{2\sqrt{2}}\bigg)
    \\=1 - Q\bigg(\frac{1}{2\sqrt{2}}\bigg)
\end{align}
%\\\textbf{Download python code from here}\\
%\begin{lstlisting}
%https://github.com/Swati-Mohanty/AI5002/blob/main/Assignment_8/codes/gate_19.py
%\end{lstlisting}
\\\textbf{Download latex code from here-}\\
\begin{lstlisting}
https://github.com/Swati-Mohanty/AI5002/blob/main/Assignment_9/codes/assignment9.tex
\end{lstlisting}

\end{document}
