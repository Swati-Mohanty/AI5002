\documentclass[journal,12pt,twocolumn]{IEEEtran}
\usepackage{amsthm}
\usepackage{graphicx}
\usepackage{mathrsfs}
\usepackage{txfonts}
\usepackage{stfloats}
\usepackage{pgfplots}
\usepackage{cite}
\usepackage{cases}
\usepackage{mathtools}
\usepackage{caption}
\usepackage{enumerate}	
\usepackage{enumitem}
\usepackage{amsmath}
\usepackage[utf8]{inputenc}
\usepackage[english]{babel}
\usepackage{multicol}
%\usepackage{xtab}
\usepackage{longtable}
\usepackage{multirow}
%\usepackage{algorithm}
%\usepackage{algpseudocode}
\usepackage{enumitem}
\usepackage{mathtools}
\usepackage{gensymb}
\usepackage{hyperref}
%\usepackage[framemethod=tikz]{mdframed}
\usepackage{listings}
    %\usepackage[latin1]{inputenc}                                 %%
    \usepackage{color}                                            %%
    \usepackage{array}                                            %%
    \usepackage{longtable}                                        %%
    \usepackage{calc}                                             %%
    \usepackage{multirow}                                         %%
    \usepackage{hhline}                                           %%
    \usepackage{ifthen}                                         %%
  \providecommand{\nCr}[2]{\,^{#1}C_{#2}}
  \providecommand{\nPr}[2]{\,^{#1}P_{#2}}
  \lstset{
%language=C,
frame=single, 
breaklines=true,
columns=fullflexible
}

\title{Assignment 3
\\Probability and Random Variables }
\author{Swati Mohanty (EE20RESCH11007) }
\date{February 2021}

\begin{document}

\maketitle


\section{Problem}
A factory has two machines A and B. Past
record shows that machine A produced
60{\%} of the items of output and machine
B produced 40{\%} of the items. Further, 2{\%}
of the items produced by machine A and
1{\%} produced by machine B were defective.
All the items are put into one stockpile and
then one item is chosen at random from this
and is found to be defective. What is the
probability that it was produced by machine B?

\section{Solution}
Let A denote the random variables of an item produced and D denote it being defective
\\(A=0) : Item produced by machine A
\\(A=1) : Item produced by machine A
\\(D=1) = Item produced is defective
\\P(A=0) = 60 {\%} = 0.6\\
P(A=1) = 40 {\%} =0.4i

\begin{align}
    P(D=1|A=0) = P(1|0) = 2 {\%} =0.02
   \\P(D=1|A=1) = P(1|1) =  1 {\%} =0.01
\end{align}
By Baye's rule,
\begin{align}
    P(A=1|D=1) = \frac{P(1)\times P(1|1)}{P(1)\times P(1|1) + P(0)\times P(1|0)}
    \\P(A=1|D=1) = \frac{0.4\times 0.01}{0.4\times 0.01 + 0.6\times 0.02}
    \\P(A=1|D=1) = 0.25
\end{align}
The probability that the defective item selected at random is produced by machine B is 25{\%}
\\Similar  result is also obtained with random samples generated using the python code.
\begin{figure}[h]
\renewcommand{\theenumi}{1}
\centering
\includegraphics[ width=\columnwidth , height =3cm]{bayes.png}
\caption{Result obtained from python code }
\label{Fig:1}
\end{figure}
\\\textbf{Download python code from here}\\
\begin{lstlisting}
https://github.com/Swati-Mohanty/AI5002/blob/main/Assignment_3/codes/bayes.py
\end{lstlisting}
\\\textbf{Download latex code from here-}\\
\begin{lstlisting}
https://github.com/Swati-Mohanty/AI5002/blob/main/Assignment_3/codes/assignment3.tex
\end{lstlisting}

\end{document}
